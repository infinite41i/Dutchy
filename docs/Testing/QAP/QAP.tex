\documentclass[12pt]{article}

\usepackage{enumerate}
\usepackage{xepersian}
\settextfont{XB Niloofar}
\begin{document}
	\begin{titlepage}
		\centering
		{\large باسمه تعالی\par}
		\vspace{1.5cm}
		{\huge \bfseries طرح تضمین کیفیت\par}
		\vspace{2cm}
		{\LARGE پروژه حسابداری دسته جمعی / دُنگی\par}
		\vfill
		{\large علی توحیدی
\par	
فرزاد شامی
\par
محسن عبداللهی

\par
علیرضا کیوانی

\par
محمد عیدی وند

\par
محمدحسین علایی

\par
امیرمحمد رضایی

\par
حامد جعفری
}
		\vfill
		{\large استاد پازوکی}
	\end{titlepage}
	\tableofcontents
	\newpage
\section{مقدمه}
\subsection{هدف}
این سند دربردارنده طرح تضمین کیفیت پروژه تهیه نرم‌افزار «حسابداری دسته جمعی/دنگی» می‌باشد. هدف از تهیه‌ی این طرح، بررسی صحت عملکرد نرم‌افزار و یافتن خطاهای احتمالی آن می‌باشد.
\subsection{دامنه‌ی کاربرد}
این طرح صرفاً به بررسی خطاهای احتمالی در نرم‌افزاری که برای اهداف آموزشی نوشته شده است می‌پردازد و با توجه به مبود زمان، اعضای تیم تعهدی در قبال رفع خطاهای یافته شده ندارند. از این رو، خروجی این آزمون و کد تولید شده برای این درس ممکن است در واقعیت قابل استفاده  یا اطمینان نباشد. از این رو مسئولیت خسارت‌های مالی و غیرمالی ناشی از استفاده از این برنامه، متوجه خود شخص کاربر است.
\subsection{تعاریف}
\begin{enumerate}[ا)]
	\item \textbf{پروژه:}
	منظور از پروژه، نرم‌افزار «حسابداری دسته جمعی/دنگی» است.
	\item \textbf{گروه:}
جمع هشت‌نفره‌ای که اسمشان در جلد این سند ذکر شد.
	\item \textbf{مورد آزمون:}
معادل فارسی 
\lr{Test Case}.
منظور از موارد آزمون، اسنادی است که فرآیند و نحوه‌ی انجام آزمون در آن به صورت گام به گام شرح داده شده است و آزمونگر، موظف است طبق آن سند، موارد را آزمایش کرده و صحت یا عدم صحت عملکرد آن را مشخص کند.
\end{enumerate}

\subsection{اختصارات}
\begin{enumerate}[ا)]
	\item \lr{\textbf{QA}} \textbf{:}
	تضمین کیفیت
\end{enumerate}
\subsection{اسناد مرتبط}
\begin{itemize}
	\item استاندارد تضمین کیفیت پروژه نظام مهندسی (نماتن)
	\item نمابرگ(اسلاید) های درس مهندسی نرم‌افزار استاد پازوکی
\end{itemize}
\section{مدیریت}
\subsection{سازمان اجرایی}
اعضای گروه عبارتند از:
\begin{itemize}
	\item علی توحیدی
	\item فرزاد شامی
	\item محسن عبداللهی
	\item علیرضا کیوانی
	\item محمد عیدی وند
	\item محمدحسین علایی
	\item امیرمحمد رضایی
	\item حامد جعفری
\end{itemize}
\subsection{فعالیت‌ها}
در فرآیند \lr{QA} پروژه، فعالیت‌های زیر انجام خواهد شد:
\subsubsection{آزمون محتوا}
در آزمون محتوا هدف کشف خطاهای املایی، نحوی، و معنایی متون و خطاهای ساختاری و مفهومی نرم‌افزار است.

برای این کار، نیاز نیست کل برنامه بررسی شود، بلکه فایل‌هایی که در موارد آزمون ذکر می‌شوند، باید بررسی شوند.
\subsubsection{آزمون رابط کاربری}
در آزمون رابط کاربری، دسترس‌پذیری راحت کاربران به تمامی امکانات نرم‌افزار و قابلیت یادگیری آسان آن سنجیده شده و صحت نمایش رابط کاربری در تمامی مرورگرها امتحان می‌شود. همچنین بررسی صحت کدهای \lr{HTML} و فرم‌ها و سایر عناصر رابط کاربری را بررسی می‌کنیم.

بخشی از این کار، به وسیله‌ی مشاهده، و بخشی به کمک ابزار 
\lr{W3C HTML Validator}
انجام می‌شود. نحوه‌ی انجام کار در موارد آزمون ذکر خواهد شد.
\subsubsection{آزمون پیمایش}
بررسی لینک‌ها، تغییر مسیرها، صحت عنوان بوکمارک‌ها، نقشه‌ی سایت، موتورهای جستجوی داخلی، و سناریوهای پیمایش نرمال و غیرنرمال(به طور مثال ورود اطلاعات غلط) در این آزمون بررسی می‌شوند. البته همه‌ی موارد ذکر شده مناسب برنامه‌ی تولید شده نمی‌باشند؛ از این رو جزئیات این آزمون در موارد آزمون ذکر خواهد شد.

\subsubsection{آزمون درستی}
در این آزمون، صحت عملکرد سیستم و برآورده کردن نیاز کاربر بررسی خواهد شد. برای انجام این کار، چند سناریو پیشنهاد خواهد شد که می‌توان طبق آنها، یا طبق سناریوی دلخواه، برنامه را سنجید. به طور مثال، سناریوی خربد یک هفته‌ی یک اتاق خوابگاه.

این سناریوها در موارد آزمون خواهد آمد.
\subsubsection{آزمون امنیت}
هدف از آزمون‌های امنیتی کشف شکاف‌های امنیتی احتمالی در سمت کاربر یا میزبان نرم‌افزار است که می‌توانند برای هر دو سمت آسیب‌زا باشند. به عنوان مثال، مشکلات امنیتی همچون \lr{CSRF}، \lr{XSS}، \lr{SQL Injection} و آپلود فایل مدیریت نشده.

اما از آنجا که برای توسعه ی این نرم افزار از فریم‌ورک جنگو استفاده شده است، و فریم‌ورک‌ها خودشان مسائل امنیتی را مدیریت می‌کنند و نرم‌افزار در دست توسعه نیازهای امنیتی چندانی ندارد،
\textbf{از این آزمون‌ها صرف نظر می‌شود.}

\section{مستندسازی}
فهرست و مشخصات مستنداتی که در جریان پروژه تولید خواهد شد، در این بخش ارائه خواهد شد.
\subsection{سند طرح تضمین کیفیت}
طرح تضمین کیفیت سندی است که هم‌اکنون در حال مطالعه‌ی آن هستید. هدف از ارائه‌ی این سند، مستندسازی فعالیت‌هایی است که برای تضمین کیفیت پروژه انجام خواهد شد.

\subsection{برگه‌های \lr{test case} و گزارش نتیجه}
این اسناد نقشه‌ی اجرای تک تک آزمون‌ها را مشخص کرده و همچنین دارای بخشی برای نتیجه‌ی این آزمون‌ها می‌باشند که توسط آزمونگر تکمیل خواهد شد.

\section{استانداردها}
استانداردهای فنی زیر در اجرای پروژه مورد استفاده قرار خواهد گرفت . رعایت این استانداردها در کلیه مراحل اجرای پروژه ضروری خواهد بود.
\begin{itemize}
	\item
	استاندارد طرح کیفیت پروژه، از مجموعه استانداردهای نظام مهندسی نرم‌ا‌فزار 
	(نماتن)، به شناسه
	\lr{NMTN.STD.QAP}
\item
	استاندارد طرح آزمون نرمافزار، از مجموعه استانداردهای نظام مهندسی نرم‌افزار 
	(نماتن)، به شناسه 
	\lr{NMTN.STD.TestPlan}
\end{itemize}

\section{ابزارها}
\begin{itemize}
\item \lr{\textbf{W3C HTML Validator}}
%\item \lr{\textbf{LambdaTest}}
%\item \lr{\textbf{Apache JMeter}}
%\item \lr{\textbf{Blazemeter}}
\end{itemize}
\end{document}